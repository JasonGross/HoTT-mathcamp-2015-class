\documentclass{article}

\usepackage{fancyhdr}
\usepackage{hyperref}
\usepackage{pageslts}
\usepackage{graphicx}
\usepackage{amsmath}
\usepackage{amssymb}

\newcommand{\chili}{\includegraphics[height=1ex]{chili}}

\lhead{Jason Gross}
\chead{Mathcamp Mentor Application}
\rhead{\today}
\cfoot{\thepage\space of \lastpageref{LastPage}}
\pagestyle{fancy}

\begin{document}
\pagenumbering{arabic}
\subsubsection*{Class: Exploring equality via homotopy and proof assistants\texorpdfstring{ (\chili\chili\,--\chili\chili\chili)}{}}
%- Class description (0.5--1 pages)
%  - Prompt
%    - Describe, in a half-page to a page, a class or independent student project that you would like to run at Mathcamp. Feel free to design your dream class or ideal project. Make sure to include some details, like difficulty, pacing, problems you might assign, etc. Keep in mind that these are bright high school students who can accomplish a lot -- but have limited time, and varying mathematical backgrounds. Some of them may meet modular arithmetic for the very first time at camp (but will be quick to pick it up and run with it); a few will have already taken classes at college, and will happily follow a development of Stokes' Theorem on manifolds; most fall somewhere in between. What we are most looking for is a coherent, interesting curriculum that demonstrates your creativity as a teacher.
\noindent \textbf{Blurb}: What does it mean for two things to be equal?  What if the things are themselves proofs of equality?  Homotopy type theory, and exciting new branch of mathematics which could replace set theory as the foundation of mathematics, has recently provided a new way of looking at equality proofs as paths in a topological space.  In this class, we will explore the nature of equality using Coq, an interactive theorem prover.

\noindent\textbf{Prerequisites}: Proof by induction, formal logic.  Optionally, programming.

\noindent\textbf{Syllabus Outline}:

\begin{itemize}
  \item
    Guided discussion --- How do we use equality?  What properties should equality have?  What properties should it not have?
    \begin{itemize}
      \item reflexivity, symmetry, transitivity (equivalence relations)
        %- but not for NaN in floating point in computers
        %  - This was probably a mistake, but shows that it's something we have to think about
      %- symmetry
      %- transitivity
      %  - sometimes; what about approximately equal?
      \item substitution
        %- Except +0 = -0, but +inf != -inf in floating point
      \item isomorphism
      \item How do we prove equality?
      \begin{itemize}
        \item by transitivity% (each equal to something we already know)
        , symmetry, reflexivity
        %- by symmetry
        %- by reflexivity
        \item by computation rules, or other axioms, maybe with substitution ($n + n = 2 \times n$; or $2 \times 3 = 6$)
        \item by induction (e.g., on natural numbers, showing $+$ commutes)
        \item by functional extensionality ($(\forall x, f(x) = g(x)) \to f = g$)
      \end{itemize}
      \item What do we do with proofs of equality?  (One answer: substitution)
      \item Parting question: When are two proofs of equality the same? (e.g., proof that $x + y = y + x$ by induction on $x$ then on $y$, vs.~on $y$ then on $x$, vs.~first apply symmetry then do double induction)
      \item Homework
      \begin{itemize}
        \item Think about when proofs of equality are the same, try to prove things in two ways related ways, and see if you can prove that the proofs are equal.
        \begin{itemize}
          \item $\forall x\ y, x + y = y + x$
          \item $\forall x\ y\ z, (x + y) + z = x + (y + z)$
        \end{itemize}
        \item Stare at J rule \\
          (\texttt{$\forall$~(A~:~Type) (x~:~A) (P~:~$\forall$ y, x = y $\to$ Type), \\ P x refl $\to$ $\forall$~y (H~:~x = y), P y H}) \\
          and K rule \\
          (\texttt{$\forall$~(A~:~Type) (x~:~A) (P~:~x = x $\to$ Type), \\ P refl $\to$ $\forall$~(H~:~x~=~x), P H}), \\
          say what they mean in words
      \end{itemize}
    \end{itemize}
    \item Exploring the J rule and the K rule in a proof assistant
    \begin{itemize}
      \item Introduce Coq, how to prove things by induction (lecture and demo)
      \begin{itemize}
        \item simplification by computation
        \item reflexivity of equality
        \item induction on natural numbers to prove addition is commutative, associative (students attempt themselves)
      \end{itemize}
      \item Introduce definition of equality in Coq (lecture)
      \begin{itemize}
        \item explain syntactic equality, computation
      \end{itemize}
      \item eliminator (J rule) (in-class exercises)
      \begin{itemize}
        \item prove symmetry, transitivity
        \item Prove \texttt{sym (sym p) = p}
        \item Prove \texttt{trans refl p = p}
        \item Prove \texttt{trans p refl = p}
      \end{itemize}
      \item Homework
      \begin{itemize}
        \item Prove UIP (uniqueness of identity proofs, \texttt{$\forall$ A (x~:~A) (p q~:~x = x), p = q}) from K
        \item Explore what else you can prove about equality in general.
        \begin{itemize}
          \item Can you prove K (or UIP) from J?
          \item relation between \texttt{sym} and \texttt{trans}?
          \item how many proofs of symmetry are there? (up to equality?)
          \item \ldots\space of transitivity?
          \item can you relate proof of \texttt{sym (sym p) = p} to other things?
        \end{itemize}
      \end{itemize}
    \end{itemize}
    \item Univalence --- isomorphism (equivalence) and equality
    \begin{itemize}
      \item Define equivalence ($\simeq$) for students (by bi-invertible map: $f : A \to B$ is an equivalence if we have $g, h : B \to A$ and $g \circ f = 1$ and $f \circ h = 1$)
      \item Proofs about equivalence (student-guided)
      \begin{itemize}
        \item all proofs that $f$ is an equivalence are equal
        \item implies quasi-inverse ($g$ such that $f \circ g = 1$ and $g \circ f = 1$, with no relation between the proofs)
        \item reflexivity
        \item symmetry
        \item transitivity
      \end{itemize}
      \item Describe univalence (\texttt{idtoequiv}$ : x = y \to x \simeq y$ is an equivalence)
      \begin{itemize}
        \item how many proofs are there of bool = bool?
      \end{itemize}
    \end{itemize}
    \item Inductive types and their equalities (possibly lecture, or possibly guided exploration individually or in small groups)
    \begin{itemize}
      \item decidable equality $\to$ UIP
      \begin{itemize}
        \item natural numbers
        \item booleans
      \end{itemize}
      \item pattern matching and induction as fundamental
      \item disjoint union/sum types
      \item cartesian product/sigma types
      \item function types/pi types
      \item classify equalities up to equivalence
      \item interesting puzzle (homework problem?): The type \texttt{\{ x : A | y = x \}} is contractible (it has an inhabitant, and all inhabitants are provably equal), even though the type \texttt{x = x} isn't.  But equals are interchangeable, so why aren't all proofs of \texttt{x = x} equal?
    \end{itemize}
    \item Higher inductive types --- custom equalities
    \begin{itemize}
      \item Define the interval (two points and a path (proof of equality) between them)
      \begin{itemize}
        \item Challenge: prove functional extensionality from the interval
      \end{itemize}
      \item Define truncation
      \item Define the circle
      \begin{itemize}
        \item Challenge: prove that the truncation of the type \texttt{base = base} is isomorphic to the integers (i.e., $\pi_1(S^1) \simeq \mathbb{Z}$)
      \end{itemize}
      \item Further exploration: truncation types, homotopies, axioms of choice, laws of excluded middle
  \end{itemize}
\end{itemize}

\end{document}