\documentclass{article}

\usepackage{fancyhdr}
\usepackage{hyperref}
%\usepackage{pageslts}
\usepackage{graphicx}
\usepackage{amsmath}
\usepackage{amssymb}
\usepackage{amsthm}

% interitemtext.sty
%
% For inserting text between \item's in a list environment
%
% Suggested by Michel Bovani.
% <http://www.esm.psu.edu/mac-tex/MacOSX-TeX-Digests/2004/MacOSX- TeX_Digest_07-12-04.html>
%
% Useage:
%
% \item This is an item.
%
% \interitemtext{This is some text not part of an item.}
%
% \item This is another item.

\makeatletter
\newcommand{\interitemtext}[1]{%
\begin{list}{}
{\itemindent 0mm\labelsep 0mm
\labelwidth 0mm\leftmargin 0mm
\addtolength{\leftmargin}{-\@totalleftmargin}}
\item #1
\end{list}}
\makeatother

\newcommand{\chili}{\includegraphics[height=1ex]{chili}}

\lhead{Exploring equality} % via homotopy and proof assistants}
\chead{Homework}
\rhead{}
\cfoot{}
\pagestyle{fancy}

\begin{document}
\rhead{Day 1}
\cfoot{(more on back)}
\section*{Day 1}

\begin{enumerate}
  \item[0.]
    If you plan to use your laptop in class tomorrow instead of the lab computers, please install Coq on it tonight from \url{https://coq.inria.fr/coq-85}.  If you need help doing this, come find me during TAU, or afterwards.
\end{enumerate}

\noindent Recall that equality has the following properties, which you may take as given:
\begin{itemize}
  \item $x = x$ (reflexivity)
  \item if $x = y$, then $y = x$ (symmetry)
  \item if $x = y$ and $y = z$, then $x = z$ (transitivity)
  \item if $x = y$, then for any function $f$, $f(x) = f(y)$ (substitution)
\end{itemize}
\subsection*{Proofs about arithmetic}
Suppose you are given a function $p : \mathbb N \times \mathbb N \to \mathbb N$ which takes in two natural numbers and outputs another natural number, and you are told that:
\begin{itemize}
  \item $p(x, 0) = x$ for all $x$, and
  \item $p(x, 1 + y) = 1 + p(x, y)$ for all $x$ and $y$.
\end{itemize}
\begin{enumerate}
  \item
    Prove by induction that for all natural numbers $x$, $p(x, 0) = p(0, x)$.
  \item
    Prove by induction that for all natural numbers $x$, $p(x, 1) = p(1, x)$.
  \item
    Prove by induction, first on $x$ and then on $y$, that for all natural numbers $x$ and $y$, $p(x, y) = p(y, x)$.
  \item
    Now prove by inducting first on $y$ and then on $x$, that for all natural numbers $x$ and $y$, $p(x, y) = p(y, x)$.
\interitemtext{You have now proved the statement $\forall x\ y,\ p(x, y) = p(y, x)$ in two different ways.  We now ask if these proofs are equal.  If you have some trouble with these problems, you may want to play with problems on the reverse side of this page before coming back to these.}
  \item
    Plug in $x = 0$ and $y = 0$ into your two proofs that $p(x, y) = p(y, x)$.  You can do this by literally re-writing your proofs, replacing $x$ and $y$ with 0 in the re-written versions.  After eliminating impossible cases (i.e., erasing parts of the proof that start with statements like ``Suppose $0 = 1 + k$ for some natural number $k$.''), how can your two proofs be said to be the same?
  \item
    Now plug in $x = 1$ and $y = 0$ into the two proofs.  Can they be said to be the same proof, for these particular numbers?
  \item
    \textbf{Challenge:} Prove, by induction, that your two proofs are the same, for all $x$ and $y$.
  \item
    \textbf{For more practice:} Find three ways to prove that $p(x, p(y, z)) = p(p(x, y), z)$, and prove that they are all equal.
  \item
    \textbf{For more practice:} Go through the above exercises again, this time with the function $m$ specified by $m(x, 0) = 0$ and $m(x, 1 + y) = x + m(x, y)$.  
\interitemtext{\subsection*{Proofs about equality more generally}}\
\interitemtext{
The \emph{\texttt{J} rule} states, formally:
\begin{center}
\texttt{$\forall$~(A~:~Type) (x~:~A) (P~:~($\forall$ (y~:~A), x = y $\to$ Type)), \\ P(x, refl$_{\texttt{x}}$) $\to$ $\forall$~(y~:~A) (H~:~x = y), P(y, H)}
\end{center}
The \emph{\texttt{K} rule} states, formally:
\begin{center}
(\texttt{$\forall$~(A~:~Type) (x~:~A) (P~:~x = x $\to$ Type), \\ P(refl$_{\texttt{x}}$) $\to$ $\forall$~(H~:~x~=~x), P(H)})
\end{center}
The notation \texttt{Type} can be interpreted as denoting, roughly, the set of all sets\footnote{There are subtleties required to avoid Cantor's paradox, but they are not relevant right now; come find me during TAU to ask about this if you are curious.}.  The $\rightarrow$ means either a function type, or logical implication.  %\footnote{Secretly, they are the same.  Come find me during TAU if you want to know more.}
The notation \texttt{x~:~A} means that \texttt{x} is an \emph{inhabitant} or \emph{element} of the set (or type) \texttt{A}.  The notation \texttt{P(y,~H)} means that you are passing the function \texttt{P} two arguments, one called \texttt{y}, and one called \texttt{H}.  The notation \texttt{refl}$_{\texttt{x}}$ is the reflexivity proof that \texttt{x~=~x}.
}
  \item
    Stare at the \texttt{J} rule and the \texttt{K} rule, and explain each of them in words.
  \item
    Here is an informal proof that the \texttt{J} rule implies that all proofs of $x = y$ are themselves equal: \\
    The \texttt{J} rule says informally that if you have a proof $H$ of $x = y$, it suffices to assume that $y$ \emph{is} $x$ (to substitute $x$ for $y$ in what you are trying to prove), and to assume that $H$ is \texttt{refl}$_x$ (to substitute \texttt{refl}$_x$ for $H$ in what you are trying to prove).  Suppose we have a type $A$, two inhabitants $x$ and $y$ of type $A$, and two proofs $q$ and $r$ that $x = y$.  By J, it suffices to assume that $y$ is $x$, that $q$ is \texttt{refl}$_x$, and hence it suffices to prove that $\texttt{refl}_x = r$, where $r$ now has type $x = x$.  To prove this, again by J, it suffices to assume that $x$ is $x$ (it already is) and that $r$ is \texttt{refl}$_x$, and hence it suffices to prove $\texttt{refl}_x = \texttt{refl}_x$.  We can prove this by \texttt{refl}$_{\texttt{refl}_x}$.  Thus for any proofs $q$ and $r$ that $x = y$, we have $q = r$. \qed \\ \\
    This proof does not, in fact, work.  What went wrong? \\ \\
    \textbf{Hint 1:} It may help to write out the arguments to \texttt{J} explicitly each time it is used informally. \\ \\
    \textbf{Hint 2:} It may help to annotate each equality with the type of the two things being compared; recall that the statement $a = b$ is only a valid type (a valid assertion) if $a$ and $b$ have the same type.
\end{enumerate}
\cfoot{}
%\cleardoublepage

%\rhead{Day 2}
%\cfoot{(more on back)}
%\section*{Day 2}


\end{document}