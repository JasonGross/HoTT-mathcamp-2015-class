\documentclass{article}

\usepackage{fancyhdr}
\usepackage{hyperref}
\usepackage{pageslts}
\usepackage{graphicx}
\usepackage{amsmath}
\usepackage{amssymb}
\usepackage{amsthm}

\newcommand{\chili}{\includegraphics[height=1ex]{chili}}

\lhead{Jason Gross}
\chead{Mathcamp Mentor Classes}
\rhead{\today}
\cfoot{\thepage\space of \lastpageref{LastPage}}
\pagestyle{fancy}

\begin{document}
\pagenumbering{arabic}
\section*{Class: Exploring equality via homotopy and proof assistants\texorpdfstring{ (\chili\chili\chili)}{}}
%- Class description (0.5--1 pages)
%  - Prompt
%    - Describe, in a half-page to a page, a class or independent student project that you would like to run at Mathcamp. Feel free to design your dream class or ideal project. Make sure to include some details, like difficulty, pacing, problems you might assign, etc. Keep in mind that these are bright high school students who can accomplish a lot -- but have limited time, and varying mathematical backgrounds. Some of them may meet modular arithmetic for the very first time at camp (but will be quick to pick it up and run with it); a few will have already taken classes at college, and will happily follow a development of Stokes' Theorem on manifolds; most fall somewhere in between. What we are most looking for is a coherent, interesting curriculum that demonstrates your creativity as a teacher.
\noindent \textbf{Possible Blurb 1}: What does it mean for two things to be equal?  What if the things are themselves proofs of equality?  Homotopy type theory, an exciting new branch of mathematics which provides an alternative to set theory as a foundation of mathematics, has recently provided a new way of looking at equality proofs as paths in a topological space.  In this class, we will explore the nature of equality using Coq, an interactive theorem prover.

\noindent \textbf{Possible Blurb 2}: Equality has recently become a HoTT topic.  By noticing and exploring the similarities between equalities and paths in topological spaces, mathematicians have discovered an alternative foundation of mathematics based on homotopy theory and type theory.  Come explore the nature of equality, learn what it takes for a proof to be computer-checkable, and experience first-hand how theorem-proving can be a game.

\noindent\textbf{Homework Recommended}

\noindent\textbf{Prerequisites}: Proof by induction, basic exposure to formal logic (comfort with \textit{modus ponens}, and the difference between axioms and theorems).  Helpful but not required: programming.

\subsection*{Syllabus Outline}

\subsubsection*{Day 1}
\begin{itemize}
  \item
    Guided discussion --- How do we use equality?  What properties should equality have?  What properties should it not have?
    \begin{itemize}
      \item reflexivity, symmetry, transitivity (equivalence relations)
        %- but not for NaN in floating point in computers
        %  - This was probably a mistake, but shows that it's something we have to think about
      %- symmetry
      %- transitivity
      %  - sometimes; what about approximately equal?
      \item substitution
        %- Except +0 = -0, but +inf != -inf in floating point
      \item isomorphism
      \item homogenous --- only valid to \emph{state} between things of the same type
      \item How do we prove equality?
      \begin{itemize}
        \item by transitivity% (each equal to something we already know)
        , symmetry, reflexivity
        %- by symmetry
        %- by reflexivity
        \item by computation rules, or other axioms, maybe with substitution ($n + n =_{\mathbb{N}} 2 \times n$; or $2 \times 3 =_{\mathbb{N}} 6$)
        \item by induction (e.g., on natural numbers, showing $x + 1 =_{\mathbb{N}} 1 + x$)
        \item by functional extensionality ($(\forall x, f(x) =_{B} g(x)) \to f =_{A \to B} g$)
        \item by isomorphism (set extensionality)
      \end{itemize}
      \item What do we do with proofs of equality?  (One answer: substitution)
    \end{itemize}
  \item
    Paths
    \begin{itemize}
      \item path reversal, constant path, concatenation
      \item substitution? (continuity)
    \end{itemize}
  \item Parting question: When are two proofs of equality the same? (e.g., proof that $x + y =_{\mathbb{N}} y + x$ by induction on $x$ then on $y$, vs.~on $y$ then on $x$, vs.~first apply symmetry then do double induction)
  \item Homework
    \begin{itemize}
      \item Think about when proofs of equality are the same, try to prove things in two ways related ways, and see if you can prove that the proofs are equal.
      \begin{itemize}
        \item $\forall x, x + 0 =_{\mathbb{N}} 0 + x$ (assuming you know that $0 + x =_{\mathbb{N}} x$ and $(1 + x) + y =_{\mathbb{N}} 1 + (x + y)$, and nothing else about natural numbers, other than induction)
        \item $\forall x\ y, x + 1 =_{\mathbb{N}} 1 + x$
        \item $\forall x\ y, x + y =_{\mathbb{N}} y + x$
        \item $\forall x\ y\ z, (x + y) + z =_{\mathbb{N}} x + (y + z)$
      \end{itemize}
      \item Stare at J rule \\
        (\texttt{$\forall$~(A~:~Type) (x~:~A) (P~:~$\forall$ y, x = y $\to$ Type), \\ P x refl $\to$ $\forall$~y (H~:~x = y), P y H}) \\
        and K rule \\
        (\texttt{$\forall$~(A~:~Type) (x~:~A) (P~:~x = x $\to$ Type), \\ P refl $\to$ $\forall$~(H~:~x~=~x), P H}), \\
        say what they mean in words
      \item
        Here is an informal proof that the J rule implies that all proofs of $x = y$ are themselves equal: \\
        The J rule says that if you have a proof $p$ of $x = y$, it suffices to assume that $y$ \emph{is} $x$ (to substitute $x$ for $y$ in what you are trying to prove), and to assume that $p$ is \texttt{refl} (to substitute \texttt{refl} for $p$ in what you are trying to prove).  Suppose we have a type $A$, two inhabitants $x$ and $y$ of type $A$, and two proofs $p$ and $q$ that $x = y$.  By J, it suffices to assume that $y$ is $x$, that $p$ is \texttt{refl}, and hence it suffices to prove that $\texttt{refl} = q$, where $q$ now has type $x = x$.  To prove this, again by J, it suffices to assume that $x$ is $x$ (it already is) and that $q$ is \texttt{refl}, and hence it suffices to prove $\texttt{refl} = \texttt{refl}$.  We can prove this by \texttt{refl}.  Thus for any proofs $p$ and $q$ that $x = y$, we have $p = q$. \qed \\
        This proof does not, in fact, work.  What went wrong? \\
        Hint 1: It may help to write out the arguments to J explicitly each time it is used informally. \\
        Hint 2: It may help to annotate each equality with the type of the two things being compared; recall that the statement $a = b$ is only a valid type (a valid assertion) if $a$ and $b$ have the same type.
    \end{itemize}
\end{itemize}

\subsubsection*{Day 2 (90 Minutes?)}
%TODO: email Ruthi and Asilata about more time -- 1.5-1.75 hours; also ask about students all having computer access during TAU
\begin{itemize}
  \item Exploring the J rule and the K rule in a proof assistant
    \begin{itemize}
      \item Introduce Coq, how to prove tautologies by function application (lecture and demo)
      \begin{itemize}
        \item various tautologies
        \item simplification by computation / judgmental equality
        \item reflexivity of equality
        \item explain syntactic equality, computation
        \item state and prove J rule by pattern matching
        \item prove symmetry by pattern matching, and separately by J
        %\item induction on natural numbers to prove addition is commutative, associative (students attempt themselves)
      \end{itemize}
      \item In-class Exercises)
      \begin{itemize}
        \item re-prove symmetry, prove transitivity
        \item Prove \texttt{sym (sym p) = p}
        \item Prove \texttt{trans refl p = p}
        \item Prove \texttt{trans p refl = p}
        \item re-do the above using the J rule explicitly
        \item See what Coq says about the informal proof that J proves UIP
        \item higher $\infty$-groupoid structure?
      \end{itemize}
      \item Homework
      \begin{itemize}
        \item Prove UIP (uniqueness of identity proofs, \texttt{$\forall$ A (x~:~A) (p q~:~x = x), p = q}) from K
        \item Explore what else you can prove about equality in general.
        \begin{itemize}
          \item Can you prove K (or UIP) from J?
          \item relation between \texttt{sym} and \texttt{trans}?
          \item how many proofs of symmetry are there? (up to equality?)
          \item \ldots\space of transitivity?
          \item can you relate proof of \texttt{sym (sym p) = p} to other things?
        \end{itemize}
      \end{itemize}
    \end{itemize}
\end{itemize}

\subsubsection*{Day 3 (90 minutes?)}
\begin{itemize}
  \item Inductive types and their equalities
    \begin{itemize}
      \item pattern matching and induction as fundamental
      \item some examples of inductive definitions (demoed in Coq)
      \begin{itemize}
        \item unit type
        \item booleans
        \item natural numbers
        \item empty type
        \item cartesian product/sigma types
        \item disjoint union/sum types
        \item function types/pi types (not inductive, but have intro and elim)
      \end{itemize}
      \item decidable equality $\to$ UIP
      \item classify equalities up to equivalence
      \item Homework
        \begin{itemize}
          \item
            interesting puzzle (homework problem?): The type \texttt{\{ x : A | y = x \}} is contractible (it has an inhabitant, and all inhabitants are provably equal), even though the type \texttt{x = x} isn't.  But equals are interchangeable, so why aren't all proofs of \texttt{x = x} equal?
          \item
            If UIP holds for a type $A$, then for any $x$ of type $A$, it also holds for the type $x = x$.  Prove this.
        \end{itemize}
      \end{itemize}
\end{itemize}

\subsubsection*{Day 4}
\begin{itemize}
  \item Univalence --- isomorphism (equivalence) and equality
    \begin{itemize}
      \item Define isomorphism ($\cong$) for students (an isomorphism is a pair of functions $f : A \to B$ and $g : B \to A$ with $g \circ f = 1$ and $f \circ g = 1$)
      \item Define equivalence ($\simeq$) for students (by bi-invertible map: $f : A \to B$ is an equivalence if we have $g, h : B \to A$ and $g \circ f = 1$ and $f \circ h = 1$)
      \item Describe univalence (\texttt{idtoequiv}$ : x = y \to x \simeq y$ is an equivalence)
      \item Proofs about equivalence (student-guided)
      \begin{itemize}
        \item all proofs that $f$ is an equivalence are equal
        \item implies quasi-inverse ($g$ such that $f \circ g = 1$ and $g \circ f = 1$, with no relation between the proofs)
        \item reflexivity
        \item symmetry
        \item transitivity
      \end{itemize}
      \begin{itemize}
        \item how many proofs are there of bool = bool?
      \end{itemize}
    \end{itemize}
\end{itemize}

\subsubsection*{Day 5 (More Advanced Topics)}
\begin{itemize}
  \item Higher inductive types --- custom equalities
    \begin{itemize}
      \item Define the interval (two points and a path (proof of equality) between them)
      \begin{itemize}
        \item Challenge: prove functional extensionality from the interval
      \end{itemize}
      \item Describe hackish way to implement HITs in Coq; mention dev version of Coq with more native HITs
      \item Define truncation
      \item Define the circle
      \begin{itemize}
        \item Re-emphasize: every function is continuous; cannot prove that all points in the circle have a path between them
        \item Challenge: prove that the truncation of the type \texttt{base = base} is isomorphic to the integers (i.e., $\pi_1(S^1) \simeq \mathbb{Z}$)
      \end{itemize}
      \item Further exploration: truncation types, homotopies, axioms of choice, laws of excluded middle
  \end{itemize}
  \item Models of type theory
  \begin{itemize}
    \item How do we know univalence is consistent?
    \item Interpretation functions
    \item Model in sets (not of univalence, though)
    \item Model in topological spaces?
    \item Model in Quillen Model Categories?
  \end{itemize}
  \item Homework --- list of open problems in HoTT, reference to the HoTT Book
\end{itemize}
\end{document}